\documentclass[12pt]{article}

\usepackage[T1,T2A]{fontenc}
\usepackage[utf8]{inputenc}
\usepackage[bulgarian]{babel}
\usepackage{graphicx}
\usepackage{amssymb}
\usepackage{amsmath}
\usepackage{commath}
\usepackage{float}
\usepackage{booktabs}   
\usepackage{ltablex}

% \usepackage{sidecap} % side captions
% \usepackage[top=1.3in, bottom=1.5in, left=1.3in, right=1.3in]{geometry}

\parindent 0px % turn off indenting

% have nicer-looking links
\usepackage{hyperref}
\hypersetup{
  colorlinks   = true, %Colours links instead of ugly boxes
  urlcolor     = red, %Colour for external hyperlinks
  linkcolor    = blue, %Colour of internal links
  citecolor   = blue
}
\usepackage[hypcap=true]{caption}

% turn off sections numbers
\makeatletter
\renewcommand{\@seccntformat}[1]{}
\makeatother


\newcommand{\JMUTitle}[9]{

  \thispagestyle{empty}
  \vspace*{\stretch{1}}
  {\parindent0cm
  \rule{\linewidth}{.7ex}}
  \begin{flushright}
    \vspace*{\stretch{1}}
    \sffamily\bfseries\Huge
    #1\\
    \vspace*{\stretch{1}}
    \sffamily\bfseries\large
    #2\\
    \vspace*{\stretch{1}}
    \sffamily\bfseries\small
    #3
  \end{flushright}
  \rule{\linewidth}{.7ex}

  \vspace*{\stretch{1}}
  \begin{center}
    \includegraphics[width=3in]{./images/logo.png} \\
    \vspace*{\stretch{1}}
    \Large Курсов проект по \\ \textit{Представяне и моделиране на знания} \\

    \vspace*{\stretch{2}}
    \large Факултет по математика и информатика\\
    \large Софийски университет\\
    
    \vspace*{\stretch{1}}
    \large Оценен от:  #8 \\[1mm]
    
    \vspace*{\stretch{1}}
    \large #7 \\

  \end{center}
}

\newcommand*{\MyIndent}{\hspace*{7em}}

%%%%%%%%%%%%%%%%% END OF PREAMBLE %%%%%%%%%%%%%%%%



\begin{document}  

  \JMUTitle
      {Icy: Онтология за Сладоледи}
      {Симеон Христов}
      {6MI3400191}
      
      {Wirtschaftswissenschaftlichen Fakultät}  % Name der Fakultaet
      {W"urzburg 2018}                          % Ort und Jahr der Erstellung
      {Януари 2023}                              % Tag der Abgabe
      {ас. Мелания Бербатова}               % Name des Erstgutachters
      {Zweitgutachter}                          % Name des Zweitgutachters

  \clearpage

\tableofcontents

\clearpage


\section{Цел} 

    Онтологията icy представя различните концепции и обекти в областта на сладоледите - видове, вкусове и съставки. Тя може да се прилага в различни контексти, включително създаване на
    функции за търсене на сладоледи в уеб сайт, или мобилно приложение, свързано със
    сладоледи, както и разработване на препоръчващи системи.
    
    \vspace{1em}
    
    Йерархията е организирана в два аспекта - на концепти, които представляват реални обекти (т.нар. \textit{DomainPartition}) и концепти, които представляват нива на сладост (т.нар. \textit{ValuePartition}). По същество това е имплементация на широко използван шаблон за проектиране на онтологии (\textit{design patten}) дискутиран в [2].


\section{Елементи на онтологията}

    \subsection{Концепти}
        
        Следващата таблица показва различните концепти в icy, използвайки синтаксиса на \textit{DL}.

        Symbols: $\doteq$, $\leq$, $\sqsubseteq$

        \begin{tabularx}{1\textwidth}{@{}X@{}}
        \toprule
        \textbf{Концепти} \\
        \midrule
        \endhead

            \textbf{DomainThing} $\sqsubseteq$ Thing \\ \tabularnewline
        
            \textbf{Country} $\doteq$ [AND DomainThing \\
                            \MyIndent [ONE-OF Argentina Australia China Greece India \\
                                    \MyIndent Indonesia Italy New\_Zeland Philippines Spain Turkey \\
                                    \MyIndent United\_Kingdom United\_States Iran] \\
                            \tabularnewline

            \textbf{Food} $\sqsubseteq$ DomainThing \\ \tabularnewline


            
            \textbf{IceCream} $\sqsubseteq$ [AND Food \\
                                      \MyIndent [EXISTS 1 :HasBase] \\
                                      \MyIndent [EXISTS 1 :HasTopping] \\
                                      \MyIndent [ALL :HasBase IceCreamBase]] \\
                                      \tabularnewline
           
            \textbf{Argentinian} $\doteq$ [AND IceCream [ALL :HasBase [ONE-OF Egg Sugar WholeMilk]] \\
                                              \MyIndent [ALL :HasTopping NaturalToppings] \\
                                              \MyIndent [FILLS :HasCountryOfOrigin Argentina]] \\
                                              \tabularnewline
           
            \textbf{Italian} $\doteq$ [AND IceCream [ALL :HasBase [ONE-OF Egg Sugar WholeMilk]] \\
                                          \MyIndent [ALL :HasTopping NaturalToppings] \\
                                          \MyIndent [FILLS :HasCountryOfOrigin Italy]] \\
                                          \tabularnewline
            
            \textbf{NamedIceCream} $\sqsubseteq$ IceCream \\ \tabularnewline
            
            \textbf{Gelato} $\sqsubseteq$ [AND NamedIceCream [ALL :HasBase WholeMilk] \\
                                                   \MyIndent [ALL :HasTopping Mango] \\
                                                   \MyIndent [FILLS :HasCountryOfOrigin Italy]] \\
                                                   \tabularnewline
            
            \textbf{Helado} $\sqsubseteq$ [AND NamedIceCream [ALL :HasBase Egg] \\
                                                   \MyIndent [ALL :HasTopping Mango] \\
                                                   \MyIndent [FILLS :HasCountryOfOrigin Argentina]] \\
                                                   \tabularnewline

            \textbf{Mochi} $\sqsubseteq$ [AND NamedIceCream [ALL :HasBase Egg] \\
                                                  \MyIndent [ALL :HasTopping [ONE-OF BlackSesame Vanilla]] \\
                                                  \MyIndent [FILLS :HasCountryOfOrigin China]] \\
                                                  \tabularnewline

            \textbf{PopularIceCream} $\doteq$ [AND IceCream \\
                                                  \MyIndent [ALL :HasTopping [ONE-OF Chocolate Coffee Mango Strawberry Vanilla]]] \\
                                                  \tabularnewline

            \textbf{Chinese} $\doteq$ [AND PopularIceCream \\
                                                  \MyIndent [ALL :HasTopping [ONE-OF BlackSesame RedBean]] \\
                                                  \MyIndent [FILLS :HasCountryOfOrigin China]] \\
                                                  \tabularnewline

            
            \textbf{IceCreamBase} $\sqsubseteq$ Food \\
                \textbf{Cream} $\sqsubseteq$ IceCreamBase \\
                \textbf{Egg} $\sqsubseteq$ IceCreamBase \\
                \textbf{Noodle} $\sqsubseteq$ IceCreamBase \\
                \textbf{Salep} $\sqsubseteq$ IceCreamBase \\
                \textbf{Sugar} $\sqsubseteq$ IceCreamBase \\
                \textbf{Water} $\sqsubseteq$ IceCreamBase \\
                \textbf{WholeMilk} $\sqsubseteq$ IceCreamBase \\ \tabularnewline
            
            \textbf{IceCreamTopping} $\sqsubseteq$ Food \\ \tabularnewline
            
            \textbf{BeanTopping} $\sqsubseteq$ IceCreamTopping \\
                \textbf{Coffee} $\sqsubseteq$ [AND BeanTopping [ALL :HasSweetness None]] \\
                \textbf{Mastic} $\sqsubseteq$ [AND BeanTopping [ALL :HasSweetness None]] \\
                \textbf{MungBean} $\sqsubseteq$ [AND BeanTopping [ALL :HasSweetness None]] \\
                \textbf{RedBean} $\sqsubseteq$ [AND BeanTopping [ALL :HasSweetness None]] \\ \tabularnewline
            
            \textbf{FruitTopping} $\sqsubseteq$ IceCreamTopping \\
                \textbf{Apple} $\sqsubseteq$ [AND FruitTopping [ALL :HasSweetness Medium]] \\
                \textbf{Avocado} $\sqsubseteq$ [AND FruitTopping [ALL :HasSweetness Medium]] \\
                \textbf{Banana} $\sqsubseteq$ [AND FruitTopping [ALL :HasSweetness Medium]] \\
                \textbf{Durian} $\sqsubseteq$ [AND FruitTopping [ALL :HasSweetness Medium]] \\
                \textbf{Jackfruit} $\sqsubseteq$ [AND FruitTopping [ALL :HasSweetness Medium]] \\
                \textbf{Lemon} $\sqsubseteq$ [AND FruitTopping [ALL :HasSweetness Medium]] \\
                \textbf{Mango} $\sqsubseteq$ [AND FruitTopping [ALL :HasSweetness Medium]] \\
                \textbf{Strawberry} $\sqsubseteq$ [AND FruitTopping [ALL :HasSweetness Medium]] \\ \tabularnewline
            
            \textbf{HerbSpiceTopping} $\sqsubseteq$ IceCreamTopping \\
                \textbf{BlackSesame} $\sqsubseteq$ [AND HerbSpiceTopping [ALL :HasSweetness None]] \\
                \textbf{PandanusLeaves} $\sqsubseteq$ [AND HerbSpiceTopping [ALL :HasSweetness None]] \\
                \textbf{Vanilla} $\sqsubseteq$ [AND HerbSpiceTopping [ALL :HasSweetness Low]] \\ \tabularnewline
            
            \textbf{LiquidTopping} $\sqsubseteq$ IceCreamTopping \\            
                \textbf{Chocolate} $\sqsubseteq$ [AND LiquidTopping [ALL :HasSweetness High]] \\
                \textbf{CoconutMilk} $\sqsubseteq$ [AND LiquidTopping [ALL :HasSweetness Medium]] \\
                \textbf{PalmSugar} $\sqsubseteq$ [AND LiquidTopping [ALL :HasSweetness High]] \\
                \textbf{RoseWater} $\sqsubseteq$ [AND LiquidTopping [ALL :HasSweetness Medium]] \\
                \textbf{SugarSyrup} $\sqsubseteq$ [AND LiquidTopping [ALL :HasSweetness High]] \\ \tabularnewline
            
            \textbf{NutTopping} $\sqsubseteq$ IceCreamTopping \\
                \textbf{Pistachios} $\sqsubseteq$ [AND NutTopping [ALL :HasSweetness None]] \\ \tabularnewline
            
            \textbf{HardToppings} $\doteq$ [AND IceCreamTopping \\
                                      \MyIndent [ONE-OF BeanTopping HerbSpiceTopping NutTopping]] \\
            \textbf{NaturalToppings} $\doteq$ [AND IceCreamTopping [NOT LiquidTopping]] \\
            \textbf{NonSweetToppings} $\doteq$ [AND IceCreamTopping [NOT SweetToppings]] \\
            \textbf{SweetToppings} $\doteq$ [AND IceCreamTopping \\ \MyIndent [ALL :HasSweetness [ONE-OF High or Medium]]] \\\tabularnewline


            \textbf{ValuePartition} $\sqsubseteq$ Thing \\ \tabularnewline
            
            \textbf{Sweetness} $\doteq$ [AND ValuePartition [ONE-OF High Medium Low None]] \\            
            \textbf{High} $\sqsubseteq$ Sweetness \\
            \textbf{Low} $\sqsubseteq$ Sweetness \\
            \textbf{Medium} $\sqsubseteq$ Sweetness \\
            \textbf{None} $\sqsubseteq$ Sweetness \\
            \tabularnewline

        \bottomrule
        \end{tabularx}


    Уточнения за горната таблица:

        \begin{itemize} 
            \item{} [NOT $d_1$ $d_2$ \ldots $d_k$] описва тези индивиди, които не са представители на всички $d_i$;
            \item{} \textit{I}[[AND DomainThing ValuePartition]] = $\emptyset$;
            \item{} \textit{I}[[AND IceCream IceCreamTopping IceCreamBase]] = $\emptyset$;
            \item{} \textit{I}[[AND BeanTopping NutTopping FruitTopping LiquidTopping HerbSpiceTopping]] = $\emptyset$;
            \item{} \textit{I}[[AND Cream Egg Noodle WholeMilk Salep Sugar Water]] = $\emptyset$;
            \item{} \textit{I}[[AND Coffee RedBean Mastic MungBean]] = $\emptyset$;
            \item{} \textit{I}[[AND Avocado Lemon Durian Strawberry Mango Banana Jackfruit Apple]] = $\emptyset$;
            \item{} \textit{I}[[AND Vanilla BlackSesame PandanusLeaves]] = $\emptyset$;
            \item{} \textit{I}[[AND SugarSyrup CoconutMilk RoseWater Chocolate PalmSugar]] = $\emptyset$.
        \end{itemize}
    
    \subsection{Свойства}

    Следващите таблици показват наличните свойства. Представени са както релации между индивидуални обекти (т.нар. \textit{Object properties}), така и релации между индивидуални обекти и данни от определени типове (т.нар. \textit{Data properties}).
    
        \subsubsection{Свойства на обектите}
    
            \begin{table}[htbp]
                \centering
                \begin{tabularx}{\textwidth}{| X | X | X | X |}
                    \hline
                    \textbf{Домейн}          & \textbf{Свойство}     & \textbf{Рейндж}     & \textbf{Характерис- тика}         \\ \hline
                            
                    Food            & hasCountryOf Origin & Country         &           -                         \\ \hline
                    Food            & hasIngredient      & Food             & \textit{transitive}                 \\ \hline
                    Food            & hasBase            & IceCreamBase     &           -                         \\ \hline
                    Food            & hasTopping         & IceCream Topping & \textit{inverse functional}         \\ \hline
                    IceCreamTopping & hasSweetness       & Sweetness        & \textit{functional}                 \\ \hline
                    Food            & isIngredientOf     & Food             & \textit{inverseOf} hasIngredient    \\ \hline
                    IceCreamBase    & isBaseOf           & Food             & \textit{inverseOf} hasBase          \\ \hline
                    IceCreamTopping & isToppingOf        & Food             & \textit{inverseOf} hasTopping       \\ \hline
                            
                \end{tabularx}
            \end{table}
    
        \subsubsection{Свойства на данните}
    
            \begin{table}[htbp]
                \centering
                \begin{tabularx}{\textwidth}{| X | X | X | X |}
                    \hline
                    \textbf{Домейн}          & \textbf{Свойство}     & \textbf{Рейндж}     & \textbf{Характерис- тика}         \\ \hline
                            
                    IceCream            & hasCalorific ContentValue & xsd:integer         &       \textit{functional}          \\ \hline
                    IceCream            & isTraditional & xsd:bool         &       \textit{functional}          \\ \hline
                            
                \end{tabularx}
            \end{table}
    
    
    
    
    \subsection{Индивиди}
    
    
    
    
    

\section{Примери за извършване на логически извод}








\section{Извършване на класификация}






\section{Заявки към базата от знание}

\subsection{DL заявки}

TODO

\subsection{SPARQL заявки}

TODO







\section{Схема на онтологията}

TODO



\section{Бъдещо развитие}

TODO




\section{Използвани технологии}

TODO





%%%%%%%%%%%%%%%%% END OF MAIN TEXT %%%%%%%%%%%%%%%%


\listoffigures

\section{Източници}

\begin{quote}

    \begin{enumerate}
    
    \item \href{https://en.wikipedia.org/wiki/Ice_cream}{Wikipedia Article on Ice Cream}
    
    \item \href{https://www.researchgate.net/publication/272829948_A_Practical_Guide_To_Building_OWL_Ontologies_Using_Protege_4_and_CO-ODE_Tools_Edition_13}{Matthew Horridge. A Practical Guide To Building OWL Ontologies Using Protégé 4 and CO-ODE Tools Edition 1.3}

    
    
    \end{enumerate}

\end{quote}

\end{document}