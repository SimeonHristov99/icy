\documentclass[12pt]{article}

\usepackage[T1,T2A]{fontenc}
\usepackage[utf8]{inputenc}
\usepackage[bulgarian]{babel}
\usepackage{graphicx}
\usepackage{amssymb}
\usepackage{amsmath}
\usepackage{commath}
% \usepackage{sidecap} % side captions
% \usepackage[top=1.3in, bottom=1.5in, left=1.3in, right=1.3in]{geometry}

\parindent 0px % turn off indenting

% have nicer-looking links
\usepackage{hyperref}
\hypersetup{
  colorlinks   = true, %Colours links instead of ugly boxes
  urlcolor     = red, %Colour for external hyperlinks
  linkcolor    = blue, %Colour of internal links
  citecolor   = blue
}
\usepackage[all]{hypcap}

% turn off sections numbers
\makeatletter
\renewcommand{\@seccntformat}[1]{}
\makeatother


\newcommand{\JMUTitle}[9]{

  \thispagestyle{empty}
  \vspace*{\stretch{1}}
  {\parindent0cm
  \rule{\linewidth}{.7ex}}
  \begin{flushright}
    \vspace*{\stretch{1}}
    \sffamily\bfseries\Huge
    #1\\
    \vspace*{\stretch{1}}
    \sffamily\bfseries\large
    #2\\
    \vspace*{\stretch{1}}
    \sffamily\bfseries\small
    #3
  \end{flushright}
  \rule{\linewidth}{.7ex}

  \vspace*{\stretch{1}}
  \begin{center}
    \includegraphics[width=3in]{./images/logo.png} \\
    \vspace*{\stretch{1}}
    \Large Курсов проект по \\ \textit{Представяне и моделиране на знания} \\

    \vspace*{\stretch{2}}
    \large Факултет по математика и информатика\\
    \large Софийски университет\\
    
    \vspace*{\stretch{1}}
    \large Проверил:  #8 \\[1mm]
    
    \vspace*{\stretch{1}}
    \large #7 \\

  \end{center}
}



%%%%%%%%%%%%%%%%% END OF PREAMBLE %%%%%%%%%%%%%%%%



\begin{document}  

  \JMUTitle
      {Icy: Онтология за Сладоледи}
      {Симеон Христов}
      {6MI3400191}
      
      {Wirtschaftswissenschaftlichen Fakultät}  % Name der Fakultaet
      {W"urzburg 2018}                          % Ort und Jahr der Erstellung
      {Януари 2023}                              % Tag der Abgabe
      {ас. Мелания Бербатова}               % Name des Erstgutachters
      {Zweitgutachter}                          % Name des Zweitgutachters

  \clearpage

\tableofcontents

\clearpage




\section{Цел} 

Онтологията icy представя различните концепции и обекти в областта на сладоледите - видове, вкусове и съставки. Тя може да се прилага в различни контексти, включително създаване на
функции за търсене на сладоледи в уеб сайт, или мобилно приложение, свързано със
сладоледи, както и разработване на препоръчващи системи.

\vspace{1em}

Йерархията е организирана в два аспекта - на класове, които представляват реални обекти (т.нар. \textit{DomainPartition}) и класове, които представляват нива на сладост (т.нар. \textit{ValuePartition}). По същество това е имплементация на широко използван шаблон за проектиране на онтологии (\textit{design patten}) дискутиран в [2].





\section{Eлементи на онтологията}

\subsection{Класове}

\subsection{Свойства}

\subsubsection{Свойства на обектите}

TODO

\subsubsection{Свойства на данните}

TODO




\subsection{Индивиди}






\section{Примери за извършване на логически извод}








\section{Извършване на класификация}






\section{Заявки към базата от знание}

\subsection{DL заявки}

TODO

\subsection{SPARQL заявки}

TODO







\section{Схема на онтологията}

TODO



\section{Бъдещо развитие}

TODO




\section{Използвани технологии}

TODO





%%%%%%%%%%%%%%%%% END OF MAIN TEXT %%%%%%%%%%%%%%%%


\listoffigures

\section{Източници}

\begin{quote}

    \begin{enumerate}
    
    \item \href{https://en.wikipedia.org/wiki/Ice_cream}{Wikipedia Article on Ice Cream}
    
    \item \href{https://www.researchgate.net/publication/272829948_A_Practical_Guide_To_Building_OWL_Ontologies_Using_Protege_4_and_CO-ODE_Tools_Edition_13}{Matthew Horridge. A Practical Guide To Building OWL Ontologies Using Protégé 4 and CO-ODE Tools Edition 1.3}

    
    
    \end{enumerate}

\end{quote}

\end{document}